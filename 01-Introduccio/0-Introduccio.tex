% Existència
Avui dia hi ha molts llenguatges de programació,
dels quals els més flexibles amb els sistemes operatius són els intèrprets.
Ja que aquests són capaços d'executar el mateix codi
independentment del sistema operatiu.\\

Com podem veure a la referència \cite{git:cpython},
tenim accés a certs intèrprets.
Només si aquests són de codi obert, però fins i tot si el citat anteriorment és codi obert,
aquest és complicat d'entendre, entre altres perquè el projecte és molt gran (74.3MB).\\

Llavors s'ha fet aquest TFG per aconseguir un intèrpret fàcil d'entendre
i a diferència d'altres, també que treballi amb punters, poden fer anar per exemple el \textbf{malloc}.


% Objectius del treball.
L'objectiu és crear un intèrpret fàcil d'entendre i funcional.
Amb els apartats ben diferenciats i que permeti visualitzar que està fent en tot moment.

% Explica, a grans trets, què has fet en aquest treball.
A grans trets, hi ha 4 apartats diferents que s'ha tingut que fer i resoldre.
Per l'analitzador lèxic, llegeix el codi font i el transforma amb estructures de C.
Per l'analitzador sintàctic, ordena les operacions perquè a l'hora d'executar sàpiga que fer.
Per l'analitzador semàntic, comprova que el nombre d'arguments i el tipus siguin els que toquen.
Per l'execució, llegeix les instruccions i executa les ordres indicades.\\

% Organització del TFG. Indica què explicaràs a cadascun dels capítols.
Organització del TFG TODO!!!
{\color{blue}No sé a que és refereix. Sobretot perquè m'ha indicat de dir que fa cada capítol}
