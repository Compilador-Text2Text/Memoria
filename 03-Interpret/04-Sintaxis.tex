\subsection{Sintàctic}
$P$ pila
$Q$ qua, sortida desitjada
$tokens$, la entrada
\begin{algorithm}
\SetKwData{Left}{left}\SetKwData{This}{this}\SetKwData{Up}{up}
\SetKwFunction{Union}{Union}\SetKwFunction{FindCompress}{FindCompress}
\SetKwInOut{Input}{entrada}\SetKwInOut{Output}{sortida}

\KwData{tokens en notació infixa.}
\KwResult{tokens en notació polaca inversa.}
$P$ una pila buida\;
$Q$ una qua buida\;
\BlankLine
\While{Quedin tokens}
{
    $t \gets$ següent token\;
    \Switch{$t$}
    {
        \Case{Valor o Variables}{$Q$.push $t$}
        \Case(\tcp*[f]{Exemple: $\sin(x)$, (,\dots}){Funció o Obert}{$P$.push $t$}
        \Case(\tcp*[f]{Exemple: ), \}, `,'\dots}){Tancat o Separador}
        {
            \While(\tcp*[f]{$P \equiv \emptyset \Rightarrow$ ERROR}){$P$.peek $\neq$ Obert}
            {
                $Q$.push $P$.pop
            }
            \If{$t$ és Tancat}{$P$.pop}
        }
        \Case(\tcp*[f]{Exemple: +, /,\dots}){Operador}
        {
            \If{$t$ és associativa per l'esquerra}{$a \gets 1$}
            \Else{$a \gets 0$}
            \While{$P$.peek és un operador $> t - a$}{$Q$.push $P$.pop}
            $P$.push $t$
        }
        \Other{Finalitzar}
    }
}
\While(\tcp*[f]{Sí és Obert $\Rightarrow$ ERROR}){$P \neq \emptyset$}{$Q$.push $P$.pop}
\caption{Shunting-yard algorithm}\label{shuntingyardalgorithm}
\end{algorithm}
\begin{equation}
3 + 4 \times 2 \div ( 1 - 5 ) \wedge 2 \wedge 3
\label{equacio:SYAbinari}
\end{equation}

\begin{table}[H]
    \centering
\begin{tabular}{ c l r }
$t$ & \multicolumn{1}{c}{$Q$}   & \multicolumn{1}{c}{$P$}   \\\hline
$3$ & $3$   &       \\
$+$ & $3$   & $+$   \\
$4$ & $3\ 4$   & $+$   \\
$\times$ & $3\ 4$   & $\times+$   \\
$2$ & $3\ 4\ 2$   & $\times+$   \\
$\div$ & $3\ 4\ 2 \times$   & $\div+$   \\
$($ & $3\ 4\ 2 \times$   & $(\div+$   \\
$1$ & $3\ 4\ 2 \times 1$   & $(\div+$   \\
$-$ & $3\ 4\ 2 \times 1$   & $-(\div+$   \\
$5$ & $3\ 4\ 2 \times 1\ 5$   & $-(\div+$   \\
$)$ & $3\ 4\ 2 \times 1\ 5 -$   & $\div+$   \\
$\wedge$ & $3\ 4\ 2 \times 1\ 5 -$   & $\wedge\div+$   \\
$2$ & $3\ 4\ 2 \times 1\ 5 - 2$   & $\wedge\div+$   \\
$\wedge$ & $3\ 4\ 2 \times 1\ 5 - 2$   & $\wedge\wedge\div+$   \\
$3$ & $3\ 4\ 2 \times 1\ 5 - 2\ 3$   & $\wedge\wedge\div+$   \\
fi & $3\ 4\ 2 \times 1\ 5 - 2\ 3 \wedge\wedge\div+$   \\
\end{tabular}
    \caption{Pila SYA amb operacions binàries de l'equació \ref{equacio:SYAbinari}}
    \label{tab:SYAn}
\end{table}
\begin{equation}
\sin(\max(2, 5) \div 3 \times 7)
\label{equacio:SYAfuncio}
\end{equation}

\begin{table}[H]
    \centering
\begin{tabular}{ c l r }
$t$ & \multicolumn{1}{c}{$Q$}   & \multicolumn{1}{c}{$P$}   \\\hline
$\sin$  &   & $\sin$    \\
$($  &   & $(\sin$    \\
$\max$  &   & $\max(\sin$    \\
$($  &   & $(\max(\sin$    \\
$2$  & $2$  & $(\max(\sin$    \\
$,$  & $2$  & $(\max(\sin$    \\
$5$  & $2\ 5$  & $(\max(\sin$    \\
$)$  & $2\ 5 \max$  & $(\sin$    \\
$\div$  & $2\ 5 \max$  & $\div(\sin$    \\
$3$  & $2\ 5 \max 3$  & $\div(\sin$    \\
$\times$  & $2\ 5 \max 3 \div$  & $\times(\sin$    \\
$7$  & $2\ 5 \max 3 \div 7$  & $\times(\sin$    \\
$)$  & $2\ 5 \max 3 \div 7 \times \sin$  &     \\
fi  & $2\ 5 \max 3 \div 7 \times \sin$  &     \\
\end{tabular}
\caption{Pila SYA amb la funció \ref{equacio:SYAfuncio}}
    \label{tab:SYAf}
\end{table}
Entenc que quan està operant
$ 3 + 2 + 5 \times 8 \wedge 2 \wedge 2 $
Escriure bonic, taules
o1 o2
+  +
-> ->
2  2
(c = o1 == -> ? 0: 1)
(o1 + c > o2)
(2 + 0 > 2) -> continua. Etc

\subsubsection{Salts}
Expreso la ida de salts aquí i no en Executable... potser hauria d'estar a executable aquest
apartat...
Tots els salts que efectua el codi normalment ho he resumit amb 3.
\begin{itemize}
    \item[\textbf{goto}] Sempre farà el salt.
    \item[\textbf{gotoZ}] Farà el salt només si és zero.
    \item[\textbf{gotoNZ}] Farà el salt només si no és zero.
\end{itemize}
Aquests estan restringits amb la funció que estiguin.
D'aquesta manera és impossible fer un salt entre funcions.

\subsubsection{Bucles}
Com ja em dit a (TODO), l'executable reserva un espai especial per operacions que ell no pot tractar.
Per exemple els \textbf{while}'s, \textbf{for}'s i \textbf{if}'s.

En codi:
\begin{itemize}
    \item while ( condició );
        \subitem instruccions;
    \item end-while;
\end{itemize}

Això s'ha de transformar amb gotoZ, gotoNZ.
Per a poder-ho fer, com que els salts estan en frases diferents,
tenim el problema de poder modificar els elements no modificats.
Llavors necessitarem d'una pila.
Aquesta serà la responsable de dir si és un \textbf{if}, \textbf{while}, \textbf{elseif}.

No sé molt bé com explicar-ho, sobretot d'una forma que s'entengui de forma agradable!
TODO