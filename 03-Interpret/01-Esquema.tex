\subsection{Esquema}
{\color{blue}
\begin{itemize}
    \item[Res d'implementar!] Estructura interna de tot el projecte. Lògica que faig anar.
    \item A molt alt nivell. (Parser, SYA, semàntica,...)
    \item Estructura d'objectes que faig anar per a poder executar el programa.
\end{itemize}
Dibuix de com està tot relacionat? o això tocaria al executable?
}



{\color{blue}
TODO, faig un altre capítol per explicar això?
Llavors podria començar a explicar el lèxic i no caldria començar amb l'execució.
}
\begin{figure}[H]
    \centering
    \includegraphics{pic/esquema.eps}
    \caption{Esquema del projecte}
    \label{fig:esquema}
\end{figure}

Com podem apreciar a la figura TODO,
% Parlem del main
el \textit{main} és el responsable dels arguments,
per saber tots els arguments podeu fer \textsc{-h}
on demanarà sempre per defecte el \textsc{-f} per saber quin programa executar.
Llavors veiem que el \textit{main} fa una crida al \textit{Inicialitzador}.

% Parlem de l'inicialitzador
L'\textit{Inicialitzador} rep els paràmetres del \textit{main}
i aquest fa el paper de controlador.
És qui fa les crides de forma ordenada per inicialitzar i finalitzar les funcions de sistema,
per a poder fer el lèxic, la sintaxis, la semàntica i per finalitzar al executor.
Fins i tot, si el \textit{main} l'hi demana, pot mostrar les funcions de sistema.

% Parlem de les funcions de sistema.
Les \textit{Funcions de Sistema} és el encarregat TODO memòria.
