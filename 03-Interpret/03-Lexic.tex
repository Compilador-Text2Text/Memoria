\subsection{Lèxic}

% Presentació.
És una simplificació de C,
amb moltes menys funcions,
força complicat i poc intuïtiu.
%
% Explico una mica.
El llenguatge de programació està fet de forma que,
pots inicialitzar variables, pots definir valors, pots cridar a funcions
i finalment pots fer crides a operacions binàries.
%
% Limitacions de les funcions.
Una de les grans limitacions que tenen les funcions,
bé a ser que sempre demanen parèntesis per a poder ser cridades.
% PROBLEMA
Fet que canvia radicalment el funcionament de C,
ja que aquest té moltes funcionalitats sense parèntesis.
% Exemples

Podem veure una mostra en l'exemple del codi \ref{codi:complicat}
\begin{lstlisting}[caption={Complex}, label=codi:complicat]
/* Concepte */
int a[10];

/* funcions sense parentesis en C */
a[3] = 2;

/* Equivalent amb l'interpret */
    "Sistema""*p=";         // Passa el punter cap al valor.
    "Preexecucio""(";
        "Locals""a";
        "Sistema""pi+e";    // punter int mes enter
        "Codi""Int"(0e)3;
    "Preexecucio"")";
    "Sistema""=int";        // Defineix el resultat.
    "Codi""Int"(0e)2e;
\end{lstlisting}

No està pensat com a llenguatge de programació,
sinó que està pensat com a llenguatge objecte.

TODO !!! TODO !!!


\subsubsection{My object}
Errors (v01):
\begin{itemize}
    \item És capaç de detectar si el fitxer està buit. (Lectura de fitxer)
    \item És capaç de detectar si no hi ha fitxer.
        \subitem No és un fitxer: exemples/
        \subitem Només podem interpretar fitxers.
    \item És capaç de dir-te a quina línia ha sigut l'error i que feia.
        \subitem De on venim:
        \subitem Declarant les variables globals
        \subitem Exactament:
        \subitem Per demanar declarar les variables, cal començar amb '-'
        \subitem Ha llegit: 'E' i esperat: '-'
        \subitem A la línia: 1
        \subitem Fins ara llegit:
        \subitem ""
\end{itemize}
és capaç de detectar fitx

\subsubsection{My c}
