\subsection{Executable}

{\color{blue}
\begin{itemize}
    \item >> TODO 1/instruccions.h  1/descriptor.h
    \item Orientat amb C.
        \subitem LIFO -pila- per les funcions dinàmiques.
        \subitem Posar alguna imatge per a fer-ho visual.
    \item LIFO -pila- per la memòria d'execució.
    \item Notació Polaca Inversa per executar el codi.
        \subitem El perquè
        \subitem Com s'executa
    \item Execució de paraula $\to$ switch
\end{itemize}
}
L'apartat més important del projecte,
ha sigut pensar com fer l'apartat executable.

Per a fer-ho he tingut que saber com treballa C.
Aquest cada execució és un procés.
\begin{figure}[!ht]
    \centering
    \includegraphics{pic/proces.eps}
    \caption{Procés en C}
    \label{fig:proces}
\end{figure}
Com podem veure a la figura \ref{fig:proces},
hi ha la pila,
on guarda les funcions, variables globals, arguments i
tota la informació que necessita per executar correctament les funcions.
Trobem la memòria dinàmica que és on guardarà la memòria usant la funció ``malloc".
Hi ha les variables globals i finalment el codi màquina.\\

Per la meva sort, l'espai lliure i la memòria dinàmica ho hereto de C.
D'aquesta forma és simplifica el projecte al no tenir que pensar amb reservar o alliberar memòria.\\

Llavors el problema és redueix en:
\begin{itemize}
    \item Pila (les crides de funcions).
    \item Variables globals.
    \item Codi màquina.
\end{itemize}

Però com que el projecte està fet amb C, les funcions pròpies de C
no puc guardar-ho de la mateixa forma que guardo les funcions entrades
per l'usuari. Ja que necessito funcions pròpies de C.

Aquest fet complica l'execució demanant separar el que són les funcions
de sistema amb les funcions que ha generat l'usuari.

\subsubsection{Pila}
Igual que fa C, l'execució la gestiona una pila.
Això simplifica la forma de fer crides a funcions,
permetent crear funcions recursives per exemple o simplement que
cridi altres funcions.\\

Però també té més utilitats.
Com que faig anar en memòria local (variable de funció) la pila,
permet poder crear `threads',
ja que podem generar una pila independent a l'altra.
Tot i que comparteixin el mateix espai lliure, les variables globals i fins i tot el mateix codi.

Tot i tenir aquest potencial, en aquest projecte no s'ha portat a terme.\\

La pila conté molta informació per a poder executar correctament les instruccions.


% variables, arguments i locals
També ha de tenir les variables,
en aquest cas s'ha fet que tingui en compte les variables locals de la funció
i també els arguments que li passen a la funció.

% Comptador
Per saber on està executant el codi,
he fet que cada pila tingui el seu propi comptador.
Fet que permet executar les instruccions una darrera l'altre.

% Descriptor
També ha de saber amb quin codi està relacionat,
llavors té un punter per saber quines instruccions haurà d'executar.

% Memòria d'execució
Té una memòria d'execució.
Aquesta és deguda al plantejament de com ha d'executar les instruccions.
El plantejament és com fa per exemple \textit{PostScript} o \textit{dc},
notació polaca inversa.
Llavors com ho indica TODO !!! TODO !!! Llibre, executar polac.
Sabem com executar les instruccions d'una forma molt fàcil.

% Retorn
Ha de saber on ha de retornar la informació.
Com ja s'ha dit, al executar les instruccions amb format pila,
cal saber on ha d'escriure les dades obtingues.
Per alliberar la pila (alliberar-ho en memòria),
és important que les funcions sempre acabin amb una funció de sortida.\\

% TODO MILLORES A FER
Possibles millores a fer:

% Variables i fer menys mallocs.
arguments i variables locals(canviar la memòria que demana de la pila)
memòria d'execució, fer-ho amb uns registres limitats, evitant fer uns quants ``mallocs''.

% Mallocs i funcions de sistema
Ara, per tenir les funcions de sistema, faig diferents ``mallocs'' per reservar quins arguments
i altra informació. Això implica que com més funcions de sistema tingui el llenguatge, més lent anirà ja que
demanarà més temps d'inicialització.
És podria millorar fent que tota aquesta informació no la guardes en cap objecte,
sinó que fos informació dins del propi codi, allargara la compilació, però l'execució guanyaria velocitat.

\subsubsection{Union valor}
Per simplificació, he decidit fer que les variables siguin un union.
Així tinc un vector que conté totes les variables siguin el que siguin.
I són d'accés directe. Ja que el vector és de mida regular.

% Tipus relacionat amb el valor
Directament relacionat, tenim un `enum' dels tipus.
Ho faig amb un `enum', perquè puc definir el \textbf{Tipus\_END},
i aquest em serveix per a saber si estic fent una crida dins dels tipus o fora.
I per depurar el codi em simplifica la feina.

\subsubsection{Localitzacions}
Tenim bàsicament dues localitzacions.
La que defineix valors a la memòria d'execució i les funcions.
Aquestes últimes també tenen la capacitat de només definir la memòria d'execució,
però normalment comporten més complexitat.\\

Pel que farà la definició de la memòria d'execució tenim:
per codi, la memòria d'execució agafarà les dades del codi.
Sinó serà per variables. Com pot ser arguments, variables locals o
variables globals.

I finalment per les funcions tenim:
les funcions creades per l'usuari i
les funcions que proporciona el sistema.