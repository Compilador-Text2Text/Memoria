\begin{equation}
3 + 4 \times 2 \div ( 1 - 5 ) \wedge 2 \wedge 3
\label{equacio:SYAbinari}
\end{equation}

\begin{table}[H]
    \centering
\begin{tabular}{ c l r }
$t$ & \multicolumn{1}{c}{$Q$}   & \multicolumn{1}{c}{$P$}   \\\hline
$3$ & $3$   &       \\
$+$ & $3$   & $+$   \\
$4$ & $3\ 4$   & $+$   \\
$\times$ & $3\ 4$   & $\times+$   \\
$2$ & $3\ 4\ 2$   & $\times+$   \\
$\div$ & $3\ 4\ 2 \times$   & $\div+$   \\
$($ & $3\ 4\ 2 \times$   & $(\div+$   \\
$1$ & $3\ 4\ 2 \times 1$   & $(\div+$   \\
$-$ & $3\ 4\ 2 \times 1$   & $-(\div+$   \\
$5$ & $3\ 4\ 2 \times 1\ 5$   & $-(\div+$   \\
$)$ & $3\ 4\ 2 \times 1\ 5 -$   & $\div+$   \\
$\wedge$ & $3\ 4\ 2 \times 1\ 5 -$   & $\wedge\div+$   \\
$2$ & $3\ 4\ 2 \times 1\ 5 - 2$   & $\wedge\div+$   \\
$\wedge$ & $3\ 4\ 2 \times 1\ 5 - 2$   & $\wedge\wedge\div+$   \\
$3$ & $3\ 4\ 2 \times 1\ 5 - 2\ 3$   & $\wedge\wedge\div+$   \\
fi & $3\ 4\ 2 \times 1\ 5 - 2\ 3 \wedge\wedge\div+$   \\
\end{tabular}
    \caption{Pila SYA amb operacions binàries de l'equació \ref{equacio:SYAbinari}}
    \label{tab:SYAn}
\end{table}